% $Id:$
%
\documentclass[a4paper,10pt,twoside]{report}
%
\usepackage[english,dutch,ngerman,english]{babel}
\usepackage{microtype}
\usepackage{relsize} %for \textscale{.75}
\usepackage{booktabs}
\usepackage{varioref}
\usepackage[verbose]{geometry}
%
%\usepackage{tikz}
%\usetikzlibrary{shapes.geometric,shapes.arrows,decorations.pathmorphing}
%\usetikzlibrary{matrix,chains,scopes,positioning,arrows,fit}
%
% These three are together:
\usepackage[latin1]{inputenc}
\usepackage[T1]{fontenc}
\usepackage[style=british]{csquotes}
%
\usepackage{color}
\usepackage{url}
\usepackage{sepnum}
\usepackage{appendix}
\usepackage{amssymb} %for twoheadrightarrow 
%
\usepackage{setspace}
\onehalfspacing
%
\usepackage{verbatim}
%
%http://tex.stackexchange.com/questions/18969/creating-a-zebra-effect-using-listings
\usepackage{lstlinebgrd}
%
% ---- Commands/definitions/&c.
%
%\makeatletter
% ----
\newcommand{\cmp}[1]{\texttt{#1}} %%computer (was cmd)
\newcommand{\step}[1]{step \textsf{#1}} %%step 1a->\step{1a}
\newcommand{\topn}[1]{top--\ensuremath{#1}}
\newcommand{\num}[1]{\sepnum{.}{\,}{}{#1}}
\newcommand{\faint}[1]{\color[gray]{0.4}{#1}}
\newcommand{\q}[1]{\enquote{#1}}
\newcommand{\qq}[1]{\enquote*{#1}}
\newcommand{\lf}[0]{\ \\ \noindent}
\newcommand{\lflf}[0]{\ \\ \ \\ \noindent}
\newcommand{\fv}[1]{\emph{#1}} %%feature value
\newcommand{\tv}[1]{\emph{#1}} %%target value
\newcommand{\numexpl}[2]{\ensuremath{#1_{\textrm{\footnotesize{#2}}}}}
%\newcommand{\dist}[2]{\ensuremath{\left(\shortstack[l]{{\sf #1} \\ \small{#2}}\right)}}
\newcommand{\dist}[2]{\scalebox{1.3}{\ensuremath{\frac{\sf #1}{#2}}}}
\newcommand{\inst}[1]{\boxed{\small{{\sf #1}}}} %instance
\newcommand{\targ}[1]{\boxed{\small{{\sf #1}}}} %target
\newcommand{\patt}[2]{\inst{#1}~\targ{#2}} %full pattern
%%                           after Big] \rule[-1mm]{0mm}{4mm}
\newcommand{\dsum}[2]{\ensuremath{\Big]_{#2}^{#1}}} %dist size/fsum
%\newcommand{\pattt}[3]{\inst{#1}~$\frac{{\sf #2}}{{\bf #3}}$} %full pattern
% ----
\newenvironment{varlist}[1]{%
\begin{list}{}{%
    \settowidth{\labelwidth}{\textsf{#1} }     %longest label length
%    \addtolength{\labelsep}{3ex}          %extra space between lbl/txt
    \setlength{\leftmargin}{\labelwidth}  %determine leftmargin
    \addtolength{\leftmargin}{\labelsep}  %determine leftmargin
    \setlength{\parsep}{0.5ex plus 0.2ex minus 0.2ex}
    \setlength{\itemsep}{0.3ex}
    \renewcommand{\makelabel}[1]{\textsf{##1}\hfill}}}%
{\end{list}}
%
\newcounter{exampleno}
\newcommand\theexample{\arabic{exampleno}}
\newenvironment{example}%
{\refstepcounter{exampleno}\vspace{\parsep}\noindent\begin{slshape}}%b
{\hspace{\stretch{1}\end{slshape}(\theexample\vspace{\parsep}}\hspace{\parsep}}%e
%
%
\newenvironment{technical}%
{\begin{slshape}}
{\end{slshape}}
%
\renewcommand\arraystretch{1.1}%
%
\newlength{\pdfwidth}
%%\setlength{\pdfwidth}{1.0\textwidth}
\setlength{\pdfwidth}{0.45\textwidth}
\newlength{\halfpage}
\setlength{\halfpage}{0.5\textwidth}
\newlength{\halfpagefig}
\setlength{\halfpagefig}{0.5\textwidth}
\addtolength{\halfpagefig}{-0.25cm}
\newlength{\thirdpage}
\setlength{\thirdpage}{0.33\textwidth}
%
% from http://blog.pcode.nl/2007/07/19/latex-tips-tricks/:
\renewcommand{\bfdefault}{sb}
%
\newcommand{\boxed}[1]{\fbox{#1\strut}}
% ----
\definecolor{Burgundy}{rgb}{.647,.129,.149}
\definecolor{Grey}{rgb}{.5,.5,.5}
%
\lstnewenvironment{bash}[1]%
{\lstset{aboveskip=\bigskipamount,belowskip=\bigskipamount,numbers=left,numberstyle=\tiny,language=bash,caption=#1,abovecaptionskip=\bigskipamount,basicstyle=\ttfamily\small,linebackgroundcolor={\ifodd\value{lstnumber}\color{green!25}\else\color{green!2}\fi}}}
{}
%
\lstnewenvironment{wout}[1]%
{\lstset{aboveskip=\bigskipamount,belowskip=\bigskipamount,language=bash,caption=#1,basicstyle=\ttfamily\small,linebackgroundcolor={\ifodd\value{lstnumber}\color{black!10}\else\color{black!2}\fi}}}
{}
%
\newcommand{\suchthat}{%
\mathrel{\ooalign{$\omega$\cr\kern-1pt$-$\kern-6.5pt$lp$}}}
\usepackage{accents}
\newcommand{\wlp}{\ensuremath{\accentset{p}{\omega}}}
%\newcommand{\wlp}{\underaccent{px}{\omega}}
%
\usepackage{relsize} 
\newcommand{\acro}[1]{\ifmmode\mathsmaller{#1}\else\textsmaller{#1}\fi} 

%--
%    a
\def\ai{\textsc{a.i.}}
\def\asr{\textsc{asr}}
%    b
\def\bleu{\textsc{bleu}}
%    c
\def\cpp{\textsc{c\small{++}}}
\def\cg{correct} %classification
\def\cd{in-distribution}
%    d
\def\DTI{\textsc{dutch-twente-ilk}}
\def\dti{\textsc{dutch-twente-ilk}}
\def\Dimpl{\textsc{Dimpl}}
%    e
\def\ep{\textsc{EuroParl}}
\def\eos{$<$/s$>$}
%\def\etc{\&c}
\def\etc{et cetera}
%    g
\def\gc{global context}
\def\giza{\textsc{giza\small{++}}}
\def\gr{gain ratio}
%    h
\def\hmm{\textsc{hmm}}
%    i
\def\IBo{\textsc{IB1}}
\def\ibo{\textsc{ib\small{1}}}
\def\IBt{\textsc{IB2}}
\def\ibt{\textsc{ib\small{2}}}
\def\ic{incorrect} %classification
\def\IGTree{\textsc{IGTree}}
\def\Igtree{\textsc{Igtree}}
\def\igtree{\textsc{igtree}}
\def\ilk{\textsc{ilk}}
%    k
\def\knn{$k$-\textsc{nn}}
%    l
\def\LM{\textsc{lm}}
\def\lm{language model}
\def\lms{language models}
%    m
\def\mb{memory based}
\def\Mb{Memory based}
\def\mblm{\textsc{mblm}}
\def\mbmt{\textsc{mbmt}}
\def\meteor{\textsc{meteor}}
\def\mono{monolithic}
\def\Moses{\textsc{Moses}}
\def\moses{\textsc{moses}}
\def\mrr{\textsc{mrr}}
\def\mt{\textsc{mt}}
%    n
\def\nist{\textsc{nist}}
\def\ngram{$n$-gram}
\def\Ngram{$N$-gram}
\def\nlp{natural language processing}
%    p
\def\per{\textsc{per}}
\def\pid{\textsc{pid}}
\def\pbmbmt{\textsc{pbmbmt}}
%    r
\def\rr{reciprocal rank}
%    s
\def\Smt{\textsc{Smt}}
\def\smt{\textsc{smt}}
\def\Srilm{\textsc{Srilm}}
\def\srilm{\textsc{srilm}}
\def\sos{$<$s$>$}
%    t
\def\Timbl{\textsc{Timbl}}
\def\timbl{\textsc{timbl}}
\def\tribl{\textsc{tribl}}
\def\Triblt{\textsc{Tribl\small{2}}}
\def\triblt{\textsc{tribl\small{2}}}
%    u
\def\unix{\textsc{unix}}
%    w
\def\wer{\textsc{wer}}
\def\Wopr{\textsc{Wopr}}
\def\wopr{\textsc{wopr}}
\def\wex{word expert}
\def\Wex{Word expert}
\def\wexs{word experts}
%--

% -*- tex -*-
%
\usepackage{ifthen}

\usepackage{tikz}
\usetikzlibrary{shapes,shadows,chains,shapes.multipart,arrows,positioning,fit}


\tikzstyle{inf} = [chamfered rectangle corners=north west, draw,text depth=.25ex]

\tikzstyle{lhs} = [signal, signal pointer angle=145,draw,signal to=nowhere,node
distance=0mm,text height=1.5ex,text depth=.25ex, signal from=east]

\tikzstyle{tgt} = [signal, signal pointer angle=145 , draw, signal to=nowhere,node distance=0mm,text height=1.5ex,text depth=.25ex, signal to=west,signal to=east,fill=gray!20]

\tikzstyle{rhs} = [signal, signal pointer angle=145 , draw, signal to=nowhere,node distance=0mm,text height=1.5ex,text depth=.25ex, signal from=west]

\tikzstyle{cls} = [rectangle, draw,node distance=0mm,text height=1.5ex,text depth=.25ex]
%different styles for cg, cd etc.^^

\tikzstyle{dist} = [rectangle,node distance=1mm,minimum size=6mm]

\begin{comment}
\begin{tikzpicture}
\node[lhs] (n0) {left hand side};
\node[tgt] (n1) [right=of n0] {target};
\node[rhs] (n2) [right=of n1] {foo};
\node[cls] (n3) [right=of n2] {bar};
\node[dist] (txt) [right=of n3] {\dist{the}{10}\dist{a}{9}\dist{in}{8}\dsum{1}{1}};
\end{tikzpicture}
\end{comment}



%wopr out: { W/M }{ CG/CD/IC }{ LHS }{ TARGET }{ RHS }{ CLASS }{ DISTR }
%            1       2          3      4         5       6       7
\newcommand{\wlr}[7]{\noindent\ensuremath{\begin{tikzpicture}
% first a place holder, haf the size of the WEX circle, so the
% WEX and  MONO boxes align.
\ifthenelse{\equal{#1}{w}}{
  \node (sp) [draw,fill=white,circle,inner sep=0pt,minimum size=0pt] {};
}{
  \node (sp) [draw=white,fill=white,circle,inner sep=0pt,minimum size=1.5pt] {};
}
% end if
\node[lhs] (n0) [right=of sp] {{\sf #3}};
% print dot if WEX (open dot if MONO, then we don't need (sp) ?)
\ifthenelse{\equal{#1}{w}}{
  \node (c1) at (n0.north west) [draw,fill,circle,inner sep=0pt,minimum size=3pt] {};
}{}
% end if
\node[tgt] (n1) [right=of n0] {{\sf #4}};
% RHS? If empty {}, skip, print rest
\ifthenelse{\equal{#5}{}}{
  \node[cls] (n3) [right=of n1] {{\sf #6}};
}{ % We do have a RHS
  \node[rhs] (n2) [right=of n1] {{\sf #5}};
  \node[cls] (n3) [right=of n2] {{\sf #6}};
} % end if RHS
% cross out if wrong answer
\ifthenelse{\equal{#2}{ic}}{
  \draw (n3.north east) -- (n3.south west);
}{}
% end if
% dotted if cd
\ifthenelse{\equal{#2}{cd}}{
  \draw [dotted, thick] (n3.north east) -- (n3.south west);
}{}
% end if
\node[dist] (txt) [right=of n3] {#7};
\end{tikzpicture}}}


% Instance
%
\newcommand{\tri}[3]{\makebox[1.5cm][l]{#1}\makebox[1.5cm][l]{#2}\makebox[1.5cm][l]{#3}}
\newcommand{\trit}[1]{\makebox[1.5cm][l]{#1}}
%
%instance l3
\newcommand{\ilt}[4]{\noindent\ensuremath{\begin{tikzpicture}
\node [rectangle, node distance=0mm,text height=1.5ex,text
depth=.25ex, draw] (i) {\tri{#1}{#2}{#3}};
\node [rectangle, node distance=0mm,text height=1.5ex,text depth=.25ex
, draw] (n1) [right=of i] {\trit{#4}};
\end{tikzpicture}}}
%
\newcommand{\bit}[1]{
\node [rectangle, node distance=0mm,text height=1.5ex,text depth=.25ex
, draw] {\bbx{#1}};
}
\newcommand{\OLDtrigram}[4]{\noindent\ensuremath{\begin{tikzpicture}
\bit{#1}\bit{#2}\bit{#3}\bit{#4}
\end{tikzpicture}}}
%
%%%
%
\newcommand{\bbx}[1]{\makebox[1.25cm][l]{\footnotesize{\sf #1}}}
\newcommand{\trigram}[4]{\noindent
\begin{tikzpicture}[start chain,node distance=1mm]
\node [on chain, drop shadow, rectangle, node distance=0mm,text height=1.5ex,text depth=.25ex
, draw, fill=white] {\bbx{#1}};
\node [on chain, drop shadow, rectangle, node distance=0mm,text height=1.5ex,text depth=.25ex
, draw, fill=white] {\bbx{#2}};
\node [on chain, drop shadow, rectangle, node distance=0mm,text height=1.5ex,text depth=.25ex
, draw, fill=white] {\bbx{#3}};
\node [on chain, drop shadow, rectangle, node distance=0mm,text height=1.5ex,text depth=.25ex
, draw, fill=gray!20] {\bbx{#4}};
\end{tikzpicture}
}


%\makeatother
%
% Chapter/section style
\usepackage{pbstyle}
% ----

%%I'm afraid that if you look at a thing long enough, it loses
%% all of its meaning. --Andy Warhol

\begin{document}

\title{Wopr}
\author{Peter Berck}
\date{\today}

\maketitle

%% ----------------------------------------------------------------
%\makeemptypage
\clearpage
\tableofcontents

%% ---

\chapter{Intro}

\Wopr{} is a wrapper around the \knn{} classifier in \Timbl{},
offering word prediction and language modeling
functionalities. Trained on a text corpus, \wopr{} can predict missing
words, report perplexities at the word level and the text level, and
generate spelling correction hypotheses.

\par
Besides the \Timbl{}-related functionality, \wopr{} can create and
manipulated data sets.

\chapter{Walk Throughs}

Without further ado, we present a number of common tasks which can be
performed with \wopr{}. We'll jump right in, and explain the different
steps and option on the way.

\section{Memory Based Language Model}

A memory based version of a trigram model. The following text corpora
are used in the experiments; \cmp{rmt.5e5} for our training data, and
\cmp{rmt.t1000} to test on. The former consists of the first
\num{500000} lines of the Reuters Newspaper corpus, while the latter
contains the last \num{1000} lines.

The following steps are typically taken when creating a model from
scratch.

\begin{varlist}{m}
\item[1] Create a lexicon
\item[2] Create a windowed data set (both for training and testing)
\item[3] Train the instance base
\item[4] Run a test on a test data set
\end{varlist}

The lexicon is created as follows.

\begin{bash}{Creating a lexicon}
wopr -r lexicon -p filename:rmt.5e5
\end{bash}

This will create a frequency list of all the tokens in the specified
file. Before we explain what we did, we will show the second step.

%\begin{lstlisting}[language=C,basicstyle=\ttfamily,numberstyle=\zebra{green!25}{green!2},numbers=left]
\begin{bash}{Creating a data set}
wopr -r window_lr -p filename:rmt.5e5,lc:2,rc:0
\end{bash}

These examples shows how \wopr{} is typically used. The two main
points to note are \cmp{-r window\_lr}, and the \cmp{-p \ldots}. The
\cmp{-r} tells \wopr{} to run the \cmp{window\_lr}
function. Everything that follows after \cmp{-p} are the parameters
passed to the function. The parameters are specified as a comma
seperated list of \cmp{keyword:value} pairs. In this case, we specify
a \cmp{filename}, \cmp{rmt.5e5} and the size of the left- and right
contexts. The context consists of two words on the left of the target,
and none on the right. This is abbreviated as \cmp{l2r0}.

\Wopr{} is quite verbose:

\begin{bash}{\Wopr{} output}
wopr -r window_lr -p filename:rmt.5e5,lc:2,rc:0
06:58:37.21: Timbl support built in.
06:58:37.22: /Users/pberck/local/
06:58:37.22: Starting wopr 1.29.3
06:58:37.22: PID:   7433 PPID:   2142
06:58:37.22: Starting.
06:58:37.22: Running: window_lr
06:58:37.22: window_lr
06:58:37.22:  filename:  rmt.5e5
06:58:37.22:  lc:        2
06:58:37.22:  rc:        0
06:58:37.22:  to:        0
06:58:37.22:  OUTPUT:    rmt.5e5.l2r0
06:59:15.61: SET filename to rmt.5e5.l2r0
06:59:15.61: Result = 0
06:59:15.61: Running for 38s
06:59:15.61: Ready.
\end{bash}

Note lines 13 and 14. In line 13, \wopr{} prints the name of the
output file it creates. These filenames are generated automatically by
\wopr{} and contain information about the parameters used. In the case
of \cmp{window\_lr}, it incorporates the context size in the file
name. This allows \wopr{} to skip steps in its processing if they have
been done before. Running the exact same command again will show the
following:

\begin{bash}{}
wopr -l -r window_lr -p filename:rmt.5e5,lc:2,rc:0
07:15:36.38: Running: window_lr
07:15:36.38: window_lr
07:15:36.38:  filename:  rmt.5e5
07:15:36.38:  lc:        2
07:15:36.38:  rc:        0
07:15:36.38:  to:        0
07:15:36.38:  OUTPUT:    rmt.5e5.l2r0
07:15:36.38: OUTPUT exists, not overwriting.
07:15:36.38: SET filename to rmt.5e5.l2r0
07:15:36.38: Result = 0
\end{bash}

Another thing to notice is the \cmp{SET} statement in line
\num{10}. This is useful when combining the different steps. This will
be explained later.

The next step is the creation of the instance base (output has been
reduced to improve readability).

\begin{bash}{Creating an instance base}
wopr -l -r make_ibase -p filename:rmt.5e5.l2r0,timbl:"-a4 +D"
10:02:16.99: Running: make_ibase
10:02:16.99: make_ibase
10:02:16.99:  timbl:     -a4 +D
10:02:16.99:  filename:  rmt.5e5.l2r0
10:02:16.99:  ibasefile: rmt.5e5.l2r0_-a4+D.ibase
...
Size of InstanceBase = 3366271 Nodes, (134650840 bytes), 29.46 % compression
Learning took 527 seconds, 732 milliseconds and 483 microseconds
Writing Instance-Base in: rmt.5e5.l2r0_-a4+D.ibase
10:12:57.87: SET ibasefile to rmt.5e5.l2r0_-a4+D.ibase
10:12:57.87: Result = 0
\end{bash}

Creating the instance based is delegated to \Timbl{}. We refer to
\cite{Daelemans+09} for a thorough explanation. The variables
specified in the \cmp{timbl} parameter are passed verbatim to \Timbl{}.

%The \cmp{\ldots} show where \Timbl{}, which is used by \wopr{} to
%train the instance base, prints its output.

And finally, we use this instance base to run a perplexity calculation
on a test set. The test set has been prepared in a step not shown
here, in the same format as the training data. It contains the last
\num{1000} lines of the Reuters data. Here we also specify the lexicon
created in step \textsf{1}. The lexicon is used to determine if a word
is known or unknown to the system.

\begin{bash}{Running a test}
wopr -l -r pplxs -p ibasefile:rmt.5e5.l2r0_-a4+D.ibase,
                    filename:rmt.t1000.l2r0,timbl:"-a4 +D",
                    lexicon:rmt.5e5.lex
...
11:50:10.93:  Correct:       3368 (20.5893)
11:50:10.93:  Correct Distr: 6250 (38.2076)
11:50:10.93:  Correct Total: 9618 (58.7969)
11:50:10.93:  Wrong:         6740 (41.2031)
11:50:10.93:  Timbl took: 11m47s
11:50:10.93:  SET px_file to rmt.t1000.l2r0_11615.px
11:50:10.93:  SET pxs_file to rmt.t1000.l2r0_11615.pxs
11:50:11.09: Result = 0
\end{bash}

\Wopr{} generates two output files. The important one is
\cmp{rmt.t1000.l2r0\_11615.px}. That contains the prediction and
statistics for each test instance we processed.

An extra perl script, \cmp{pplxs\_px.pl} has been supplied to post
process the output. It can be found in the \cmp{etc/} directory in the
\wopr{} distribution. It computes perplexity and other statistics on
the output. It is run as follows.

\begin{bash}{Post processing of \wopr{} output}
perl pplx_px.pl -f rmt.t1000.l2r0_11615.px -l2 -r0
\end{bash}

The file to process is specified with the \cmp{-f} parameter. The
context size is specified with the \cmp{-l} and \cmp{-r}
parameters. It contains a line for each instance, followed by a
summary. A fragment of the output:

\begin{wout}{}
0.02119694  -5.5600  -1.6737  1 1 [01010] ways
0.41176411  -1.2801  -0.3854  1 1 [01010] of
\end{wout}

It shows the probability of the classification, followed by the
$log_2$ and $log_{10}$ of the probability. For an explanation of the
other values, see the reference chapter. The summary at the end shows,
amongst other statistics, the perplexity on the test data.

\begin{wout}{}
Wordcount: 16358 sentencecount: 0 oovcount: 542
Wopr ppl:    251.41 Wopr ppl1:    251.41  (No oov words.)
\end{wout}

To get performance per word, the \cmp{-w} flag can be added.

\begin{wout}{}
winner: cg:0 (0.00%) cd:0 (0.00%) ic:1 (100.00%)
with: cg:11 (18.33%) cd:22 (36.67%) ic:27 (45.00%)
\end{wout}

This show that the word \cmp{winner} appeared once in the test set,
and was incorrectly predicted. The word \cmp{with} was predicted
correctly in ten cases, \etc{}.

\subsection{Combining Commands}

As we mentioned before, the \wopr{}-commands can be combined. \Wopr{}
has a mechanism to automatically generate output filenames based on
certain parameters. It will \q{create} come of these parameters after
each run, so they will be available to the next function as input
without having to specify them from the start. We can combine several
of the steps we shows in the previous part like this.

\begin{bash}{Combining commands}
wopr -r lexicon,window_lr,make_ibase,pplxs -p filename:rmt.5e5,
                                              timbl:"-a4 +D",lc:2,rc:0
...
09:11:54.15: SET ibasefile to rmt.5e5.l2r0_-a4+D.ibase
09:11:54.15: Result = 0
09:11:54.15: Running: pplxs
09:11:54.15: pplxs
09:11:54.15:  ibasefile:      rmt.5e5.l2r0_-a4+D.ibase
09:11:54.15:  lexicon:        rmt.5e5.lex
09:11:54.15:  counts:         rmt.5e5.cnt
...
\end{bash}

If we look at the last step, \cmp{pplxs}, we see that the instance base
filename and the other parameters have been taken from the previous
steps. Note that we cannot also generate the windowed test set in this
chain. That has to be done seperately.

\section{\ngram{}-model}

\Wopr{} can also create a classical \ngram{}-model. This functionality
has been included so as to be able to compare the \mb{} \lm{} to 
\q{pure} \ngram{}-model. It has neither been optimised for speed,
nor for memory efficiency.

Creation of the model consists of one step; \cmp{ngl}.

\begin{bash}{Creation of an \ngram{} model}
wopr -r ngl -p filename:rmt.5e5
11:19:40.16: Running: ngl
11:19:40.18: ngl
11:19:40.18:  filename:  rmt.5e5
11:19:40.18:  n:         3
11:19:40.18:  fco:       0
11:19:40.18:  OUTPUT:    rmt.5e5.ngl3f0
11:19:40.18: Reading...
11:21:33.15: Writing...
11:22:36.35: SET ngl to rmt.5e5.ngl3f0
11:22:45.65: Result = 0
\end{bash}

The output contains a list with all the \ngram{}s followed by a
frequency count and a conditional probability. The default \ngram{}
size is three.

The command to process a test file is called \cmp{ngt}.

\begin{bash}{testing with an \ngram{} model}
wopr -r ngt -p ngl:rmt.5e5.ngl3f0,testfile:rmt.t1000
11:31:50.24: Running: ngt
11:31:50.24: ngt
11:31:50.24:  filename:  rmt.t1000
11:31:50.24:  ngl file:  rmt.5e5.ngl3f0
11:31:50.24:  counts:    
11:31:50.24:  n:         3
11:31:50.24:  id:        
11:31:50.24:  OUTPUT:    rmt.t1000.ngt3
11:31:50.24:  OUTPUT:    rmt.t1000.ngp3
11:31:50.24: NOTICE: cannot read counts file, no smoothing will be applied.
11:31:50.24: P(0) = 1e-06
11:31:50.24: Reading ngrams...
11:31:59.71: Writing output...
11:32:00.02: Total words: 16358
11:32:00.02: Total oovs: 542
11:32:00.02: Total log2prob: -121551
11:32:00.02: Average log2prob: -7.6853
11:32:00.02: Average pplx: 205.828
11:32:00.02: SET ngt_file to rmt.t1000.ngt3
11:32:00.02: SET ngp_file to rmt.t1000.ngp3
11:32:01.90: Result = 0
\end{bash}

This produces two output files. One, \cmp{rmt.t1000.ngt3}, containing
statistics for each classifiation, as shown in the following fragment. It
shows a word, the \ngram{} probability, the size of the matching
\ngram{} and the \ngram{} itself.

\begin{wout}{}
and 0.0928793 2 puts and
calls 0.5 3 puts and calls
\end{wout}

The other, \cmp{rmt.t1000.ngp3}, contains statistics per line of text.

\begin{wout}{}
-81.9518 293.087 10 0 The officers quickly managed to calm down ...
\end{wout}

The output contains the entropy, perplexity, word count and number of
unknown words, followed by the text line itself.

\chapter{Installation}

\Timbl{} is required to run \wopr{}.

Assuming \Timbl{} is configured and installed locally with
\cmp{--prefix=/home/pberck/local}. We also install \wopr{} locally:

%\begin{lstlisting}[language=bash,basicstyle=\ttfamily,linebackgroundcolor={\ifodd\value{lstnumber}\color{green!25}\else\color{green!2}\fi}]
\begin{bash}{Configuring \wopr{} locally}
sh bootstrap
./configure --prefix=/home/pberck/local \
            --with-timbl=/home/pberck/local
make
make install
\end{bash}

If \Timbl{} is installed system wide, the \cmp{--with-timbl} parameter
can be omitted. The configure script should be able to figure out
where the libraries are. Wopr can be built without \Timbl{} by specifying
\cmp{--without-timbl}. This will leave you with a \wopr{} which can
create datafiles and run an \ngram{} model only.
%On Debian we need: \cmp{export LD\_LIBRARY\_PATH=\$LD\_LIBRARY\_PATH:/home/pberck/local/lib}
The \cmp{LD\_LIBRARY\_PATH} and \cmp{PATH} variables might need to be
set as well. A short test to see if all went well:

\begin{bash}{Running \wopr{}}
./wopr
10:44:32.19: Timbl support built in.
10:44:32.19: /Users/pberck/local/
10:44:32.19: Starting wopr 1.29.3
10:44:32.20: PID:   4173 PPID:   1136
10:44:32.20: Starting.
10:44:32.21: Running for 00s
10:44:32.21: Ready.
\end{bash}

The second line shows the path tot the \Timbl{} libraries.

\chapter{Performance}

\section{Memory}

\begin{comment}
// ---- 2011-06-13 15:47:50 Mon -----------------------------------------------
ceto
pberck@ceto:/exp2/pberck\$ /exp/pberck/wopr/wopr -l -r make_ibase -p filename:nyt.3e7.100000.l3r0,timbl:"-a1 +D"

and:
pberck@ceto:/exp/pberck/wopr/etc\$ bash watch_wopr.sh 30341 88G 10 1

gnuplot 30341_mem.plot
durian:doc pberck\$ scp -P223 pberck@portal.ticc.uvt.nl:doorgeefluik/303* .

\begin{figure}[!ht]
  \centering
  \includegraphics[width=\pdfwidth]{30341_mem.pdf}
 \caption{Example gc instance base}
  \label{fig:30341_mem.pdf}
\end{figure}

// ---- 2011-06-13 15:55:26 Mon -----------------------------------------------

pberck@ceto:/exp2/pberck\$ wc -l nyt.3e7.1000000.l3r0                            22855429 nyt.3e7.1000000.l3r0

/exp/pberck/wopr/wopr -l -r make_ibase -p filename:nyt.3e7.1000000.l3r0,timbl:"-a1 +D"
bash watch_wopr.sh 32376 88G 10 1

testing with about ibase:
/exp/pberck/wopr/wopr -l -r pplxs -p
ibasefile:nyt.3e7.1000000.l3r0_-a1+D.ibase,timbl:"-a1+D",filename:lc/nyt.tail1000.l3r0
bash watch_wopr.sh  26251 88G 10 1
\end{comment}

Memory use when training a classifier on \num{1000000} lines of text
(\num{22855429} instances).

\begin{figure}[!ht]
  \centering
  \includegraphics[width=\pdfwidth]{32376_mem.pdf}
 \caption{nyt.1000000.l3r0}
  \label{fig:32376_mem.pdf}
\end{figure}

Memory use when using the above instance base for a classification
task.

\begin{figure}[!ht]
  \centering
  \includegraphics[width=\pdfwidth]{26251_mem.pdf}
 \caption{nyt.1000000.l3r0 ibase, test tail1000}
  \label{fig:26251_mem.pdf}
\end{figure}

\chapter{Reference}

\section{Data}

\subsection{window\_lr}

\begin{varlist}{lc}
\item[lc] Left context. Number of words on the left side of the target.
\item[rc] Right context. Number of words on the right side of the target.
\item[to] target offset. Shift the target position. Setting this to
  one will skip one word on the left, that is, a b -> d, c will be skipped.
\end{varlist}

\section{Training}

\subsection{make\_ibase}

\begin{varlist}{filename}
\item[filename] Filename of the windowed data set.
\item[timbl] Settings for \Timbl{}.
\end{varlist}

\section{Testing}

\subsection{\cmp{pplxs}}

The \cmp{pplxs} function is geared towards word prediction. It takes
an instance base, and calls \Timbl{} to process the test file. A
number of word prediction related values are calculated and written to
the output file.

The output contains the following:

\begin{wout}{}
85 bp bp 0 0 1 cg k 1 1 1 11 1 [ bp 11 ]
bp over over 0 0 1 cg k 1 1 1 11 1 [ over 11 ]
over LIBOR Libor -16.4148 0 87364.5 ic k 1 1 1 11 0 [ Libor 11 ]
LIBOR delayed . -13.915 2.502 15452.2 ic k 1 1 8 44 0 [ . 14 , 11 for 6 ]
\end{wout}

The first tokens contain the instance (in this case \cmp{l1r0}),
followed by the classification. This is followed by the logprob of the
classification, the entropy of the distribution and the word level
perplexity. The logprob is the $log_2$ of the probability of the
classification. The entropy of the distribution $D$ is calculated as
follows:

\begin{equation}
H(p) = -\sum_{x \in D} \, p(x)\log_2 p(x)
\label{eq:entropy}
\end{equation}

The word level perplexity is defined as:

\begin{equation}
wlp = 2^{-p_{classification}}
\label{eq:wlp}
\end{equation}

These three numbers are followed by an indicator (cg, cd or ic),
followed by an known/unknown word indicator. This is followed by the
match-depth and a boolean value indicating matched-at-leaf (one or
zero). See \cite{Daelemans+09} for a more thorough explanation. The
following three values are the distribution size, the sum of the
distribution frequencies, and the \emph{reciprocal rank} of the
answer. Finally, the output contains the top-$n$ of the distributions,
specified as a list of \cmp{token frequency} pairs.

\begin{varlist}{ibasefile}
\item[filename] Test file to process.
\item[dir] Instead of one file, a whole directory with files will be
  processed if the \cmp{dir} parameter is specified instead of the
  \cmp{filename} parameter.
\item[dirmatch] A regular expression which determines which files will
  be read from the directory specified with \cmp{dir}.
\item[ibasefile] Filename of the instance base.
\item[timbl] Settings used to create the instance base.
\item[lexicon] Lexicon to determine known/unknown words.
\item[counts] Counts of ranks for the lexicon. This can be used to
  introduce smoothed or changed probabilities to the perplexity
  calculations. 
\item[hapax] Used when reading the lexicon.
\item[lc] Not used.
\item[rc] Not used.
\item[topn] Number of elements from the \Timbl{} distribution to
  include in the output. Most frequent one first.
\item[cache] Number of elements in the cache (the default of three is
  good). 
\item[cth] Cache threshold: A distribution is not cached until it
  contains this many items.
\item[is] Include the whole sentence in the output (pxs).
\item[id] Identifier to append to the output filenames. If not
  specified, it defaults to the \pid{}.
\end{varlist}

\subsection{\cmp{gt}}

The \cmp{gt} function is a general test routine; it calls \Timbl{} and
writes the output to a file. The word prediction calculations are not
performed. The output filename is the name of the testfile, followed
by the \cmp{id} and the suffix \cmp{gt}.

The output contains the following:

\begin{wout}{}
million million cg 1 1 20 224 [ million 139 mln 30 billion 11 ]
\end{wout}

The first token is the target, the second the classification from
\Timbl{}. This is followed by an indicator (cg, cd, or ic), followed
by two values which indicate the match-depth, and a boolean value
specifying matched-at-leaf (one or zero). If the \cmp{topn} parameter
had been specified, this is followed by the distribution count, the
sum of the frequency of the elements in the distribution, and the
top-$n$ from the distribution. The latter is a list with
\cmp{token frequency} pairs.

\begin{varlist}{ibasefile}
\item[filename] Test file to process.
\item[dir] Instead of one file, a whole directory with files will be
  processed if the \cmp{dir} parameter is specified instead of the
  \cmp{filename} parameter.
\item[dirmatch] A regular expression which determines which files will
  be read from the directory specified with \cmp{dir}.
\item[ibasefile] Filename of the instance base.
\item[timbl] Settings used to create the instance base.
\item[cache] Number of elements in the cache (the default of three is
  good). 
\item[cth] Cache threshold: A distribution is not cached until it
  contains this many items.
\item[cs] Size of the cache which caches test instances. Default is
  \num{10000}. Note that this is a different cache from the cache which
  caches distributions. 
\item[topn] Number of elements from the \Timbl{} distribution to
  include in the output. Most frequent one first.
\item[id] Identifier to append to the output filenames. If not
  specified, it defaults to the \pid{}.
\end{varlist}


\bibliographystyle{apalike}
\bibliography{ilk}

\end{document}


